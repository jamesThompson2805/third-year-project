\chapter{Strawberry}
\hypertarget{md_external_2data_2UCRArchive__2018_2Strawberry_2README}{}\label{md_external_2data_2UCRArchive__2018_2Strawberry_2README}\index{Strawberry@{Strawberry}}
\label{md_external_2data_2UCRArchive__2018_2Strawberry_2README_autotoc_md215}%
\Hypertarget{md_external_2data_2UCRArchive__2018_2Strawberry_2README_autotoc_md215}%
 Food spectrographs are used in chemometrics to classify food types, a task that has obvious applications in food safety and quality assurance. These data were processed using Fourier transform infrared (FTIR) spectroscopy with attenuated total reflectance (ATR) sampling. More details are provided in \mbox{[}1\mbox{]}\mbox{[}2\mbox{]}.

The classes are strawberry purees (authentic samples) and non-\/strawberry purees (adulterated strawberries and other fruits).

Train size\+: 613

Test size\+: 370

Missing value\+: No

Number of classses\+: 2

Time series length\+: 235

Data donated by Katherine Kemsley and Anthony Bagnall (see \mbox{[}1\mbox{]}, \mbox{[}2\mbox{]}, \mbox{[}3\mbox{]}).

\mbox{[}1\mbox{]} Holland, J. K., E. K. Kemsley, and R. H. Wilson. "{}\+Use of Fourier transform infrared spectroscopy and partial least squares regression for the detection of adulteration of strawberry purees."{} Journal of the Science of Food and Agriculture 76.\+2 (1998)\+: 263-\/269.

\mbox{[}2\mbox{]} \href{https://csr.quadram.ac.uk/example-datasets-for-download/}{\texttt{ https\+://csr.\+quadram.\+ac.\+uk/example-\/datasets-\/for-\/download/}}

\mbox{[}3\mbox{]} \href{http://www.timeseriesclassification.com/description.php?Dataset=Strawberry}{\texttt{ http\+://www.\+timeseriesclassification.\+com/description.\+php?\+Dataset=\+Strawberry}} 