\chapter{Arrow\+Head}
\hypertarget{md_external_2data_2UCRArchive__2018_2ArrowHead_2README}{}\label{md_external_2data_2UCRArchive__2018_2ArrowHead_2README}\index{ArrowHead@{ArrowHead}}
\label{md_external_2data_2UCRArchive__2018_2ArrowHead_2README_autotoc_md20}%
\Hypertarget{md_external_2data_2UCRArchive__2018_2ArrowHead_2README_autotoc_md20}%
 The Arrow\+Head data consist of outlines of the images of arrowheads. The shapes of the projectile points are converted into a time series using the angle-\/based method. The classification of projectile points is an important topic in anthropology. The classes are based on shape distinctions such as the presence and location of a notch in the arrow. The problem in the repository is a length normalised version of that used by Ye and Keogh \mbox{[}1\mbox{]}. The three classes are "{}\+Avonlea"{}, "{}\+Clovis"{} and "{}\+Mix"{}."{}  \+Train size\+: 36  \+Test size\+: 175  \+Missing value\+: No  \+Number of classses\+: 3  \+Time series length\+: 251  \+Data donated by Lexiang Ye and Eamonn Keogh (see \mbox{[}1\mbox{]}, \mbox{[}2\mbox{]}, \mbox{[}3\mbox{]}).  \mbox{[}1\mbox{]} Ye, Lexiang, and Eamonn Keogh. "{}Time series shapelets\+: a new primitive for data mining."{} Proceedings of the 15th ACM SIGKDD international conference on Knowledge discovery and data mining. ACM, 2009.

\mbox{[}2\mbox{]} \href{http://alumni.cs.ucr.edu/~lexiangy/shapelet.html}{\texttt{ http\+://alumni.\+cs.\+ucr.\+edu/\texorpdfstring{$\sim$}{\string~}lexiangy/shapelet.\+html}}

\mbox{[}3\mbox{]} \href{http://www.timeseriesclassification.com/description.php?Dataset=ArrowHead}{\texttt{ http\+://www.\+timeseriesclassification.\+com/description.\+php?\+Dataset=\+Arrow\+Head}} 