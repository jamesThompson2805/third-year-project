\chapter{Smooth\+Subspace}
\hypertarget{md_external_2data_2UCRArchive__2018_2SmoothSubspace_2README}{}\label{md_external_2data_2UCRArchive__2018_2SmoothSubspace_2README}\index{SmoothSubspace@{SmoothSubspace}}
\label{md_external_2data_2UCRArchive__2018_2SmoothSubspace_2README_autotoc_md211}%
\Hypertarget{md_external_2data_2UCRArchive__2018_2SmoothSubspace_2README_autotoc_md211}%
 The data were originally intended for testing whether a clustering algorithm is able to extract smooth subspaces for clustering time series data \mbox{[}1\mbox{]}.

There are 3 classes corresponding to which cluster the time series belong to. Each time series contain a continuous subspace spanning over 5 continuous time stamps.
\begin{DoxyItemize}
\item For cluster 1, it is from time stamp 1-\/5
\item For cluster 2, it is from time stamp 6-\/10
\item For cluster 3, it is from time stamp 11-\/15.
\end{DoxyItemize}

The rest of the time series are randomly generated.

Train size\+: 150

Test size\+: 150

Missing value\+: No

Number of classses\+: 3

Time series length\+: 15

There is nothing to infer from the order of examples in the train and test set.

Data created by Xiaohui Huang et al. (see \mbox{[}1\mbox{]}). Data edited by Hoang Anh Dau.

\mbox{[}1\mbox{]} Huang, Xiaohui, et al. "{}\+Time series k-\/means\+: A new k-\/means type smooth subspace clustering for time series data."{} Information Sciences 367 (2016)\+: 1-\/13. 