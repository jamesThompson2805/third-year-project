\chapter{Meat}
\hypertarget{md_external_2data_2UCRArchive__2018_2Meat_2README}{}\label{md_external_2data_2UCRArchive__2018_2Meat_2README}\index{Meat@{Meat}}
\label{md_external_2data_2UCRArchive__2018_2Meat_2README_autotoc_md137}%
\Hypertarget{md_external_2data_2UCRArchive__2018_2Meat_2README_autotoc_md137}%
 Food spectrographs are used in chemometrics to classify food types, a task that has obvious applications in food safety and quality assurance. The classes are chicken, pork and turkey. Duplicate acquisitions are taken from 60 independent samples obtained using Fourier transform infrared (FTIR) spectroscopy with attenuated total reflectance (ATR) sampling. The data are described in more detail in \mbox{[}1\mbox{]}.

Train size\+: 60

Test size\+: 60

Missing value\+: No

Number of classses\+: 3

Time series length\+: 448

Data donated by Katherine Kemsley and Anthony Bagnall (see \mbox{[}1\mbox{]}, \mbox{[}2\mbox{]}).

\mbox{[}1\mbox{]} Al-\/\+Jowder, O., E. K. Kemsley, and R. H. Wilson. "{}\+Mid-\/infrared spectroscopy and authenticity problems in selected meats\+: a feasibility study."{} Food Chemistry 59.\+2 (1997)\+: 195-\/201.

\mbox{[}2\mbox{]} \href{http://www.timeseriesclassification.com/description.php?Dataset=Meat}{\texttt{ http\+://www.\+timeseriesclassification.\+com/description.\+php?\+Dataset=\+Meat}} 