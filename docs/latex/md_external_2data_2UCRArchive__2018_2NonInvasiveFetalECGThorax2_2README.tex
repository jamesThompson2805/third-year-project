\chapter{Non\+Invasive\+Fatal\+ECGThorax2}
\hypertarget{md_external_2data_2UCRArchive__2018_2NonInvasiveFetalECGThorax2_2README}{}\label{md_external_2data_2UCRArchive__2018_2NonInvasiveFetalECGThorax2_2README}\index{NonInvasiveFatalECGThorax2@{NonInvasiveFatalECGThorax2}}
\label{md_external_2data_2UCRArchive__2018_2NonInvasiveFetalECGThorax2_2README_autotoc_md157}%
\Hypertarget{md_external_2data_2UCRArchive__2018_2NonInvasiveFetalECGThorax2_2README_autotoc_md157}%
 Since the late 19th century, decelerations of fetal heart rate have been known to be associated with fetal distress. Intermittent observations of fetal heart sounds (auscultation) became standard clinical practice by the mid-\/20th century. The most accurate method for measuring fetal heart rate is direct fetal electrocardiographic (FECG) monitoring using a fetal scalp electrode. This is possible only in labor, however, and is not common in current clinical practice because of its associated risks. Noninvasive FECG monitoring makes use of electrodes placed on the mother\textquotesingle{}s abdomen.

Each time series corresponds to a record of ECG from the left and right thorax. The original data are from \mbox{[}1\mbox{]} and later re-\/formatted for time series classification by Hu and Keogh.

Train size\+: 1800

Test size\+: 1965

Missing value\+: No

Number of classses\+: 42

Time series length\+: 750

Data donated by Bing Hu and Eamonn Keogh (see \mbox{[}1\mbox{]}, \mbox{[}2\mbox{]}).

\mbox{[}1\mbox{]} \href{https://www.physionet.org/challenge/2013/}{\texttt{ https\+://www.\+physionet.\+org/challenge/2013/}}

\mbox{[}2\mbox{]} \href{http://www.timeseriesclassification.com/description.php?Dataset=NonInvasiveFatalECGThorax2}{\texttt{ http\+://www.\+timeseriesclassification.\+com/description.\+php?\+Dataset=\+Non\+Invasive\+Fatal\+ECGThorax2}} 