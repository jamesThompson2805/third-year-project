\chapter{Herring}
\hypertarget{md_external_2data_2UCRArchive__2018_2Herring_2README}{}\label{md_external_2data_2UCRArchive__2018_2Herring_2README}\index{Herring@{Herring}}
\label{md_external_2data_2UCRArchive__2018_2Herring_2README_autotoc_md119}%
\Hypertarget{md_external_2data_2UCRArchive__2018_2Herring_2README_autotoc_md119}%
 Otoliths are calcium carbonate structures present in many vertebrates, found within the sacculus of the pars inferior. There are three types of otoliths\+: sagittae, lapilli, and asterisci. In fish, it is primarily the sagittal otoliths that are studied, as they are larger and easier to prepare and observe. Otoliths vary markedly in shape and size between species, but are of similar shape to other stocks of the same species. Otoliths contain information that can be used by expert readers to determine several key factors important in managing fish stock. Analysis of otolith boundaries may allow estimation of stock composition, including whether the samples are from one stock or multiple stocks, allowing management decisions to be made. These data consist of Otholith outlines from two classes of herring \mbox{[}1\mbox{]}\+: North sea or Thames. More details are in the paper by Mapp et al. \mbox{[}2\mbox{]}.

Train size\+: 64

Test size\+: 64

Missing value\+: No

Number of classses\+: 2

Time series length\+: 512

Data donated by James Mapp and Anthony Bagnall (see \mbox{[}2\mbox{]}, \mbox{[}3\mbox{]}).

\mbox{[}1\mbox{]} \href{https://en.wikipedia.org/wiki/Herring}{\texttt{ https\+://en.\+wikipedia.\+org/wiki/\+Herring}}

\mbox{[}2\mbox{]} Mapp, J., et al. "{}\+Clupea Harengus\+: Intraspecies Distinction Using Curvature Scale Space and Shapelets."{} Proc. 2nd International Conference on Pattern Recognition Applications and Methods (ICPRAM 2013). 2013.

\mbox{[}3\mbox{]} \href{http://www.timeseriesclassification.com/description.php?Dataset=Herring}{\texttt{ http\+://www.\+timeseriesclassification.\+com/description.\+php?\+Dataset=\+Herring}} 