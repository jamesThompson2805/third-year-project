\chapter{FordB}
\hypertarget{md_external_2data_2UCRArchive__2018_2FordB_2README}{}\label{md_external_2data_2UCRArchive__2018_2FordB_2README}\index{FordB@{FordB}}
\label{md_external_2data_2UCRArchive__2018_2FordB_2README_autotoc_md75}%
\Hypertarget{md_external_2data_2UCRArchive__2018_2FordB_2README_autotoc_md75}%
 These data were originally used in a competition in the 2008 IEEE World Congress on Computational Intelligence. The classification problem is to diagnose whether a certain symptom exists or not in an automotive subsystem. Each case consists of 500 measurements of engine noise.

There are two related problems, {\itshape FordA} and {\itshape FordB}. For {\itshape FordA}, both the train and test data were collected in typical operating conditions with minimal noise contamination. For {\itshape FordB}, the train data were collected in typical operating conditions, but the test data were collected under noisy conditions.

Train size\+: 3636

Test size\+: 810

Missing value\+: No

Number of classses\+: 2

Time series length\+: 500

Data donated by Anthony Bagnall (see \mbox{[}1\mbox{]}).

\mbox{[}1\mbox{]} \href{http://www.timeseriesclassification.com/description.php?Dataset=FordB}{\texttt{ http\+://www.\+timeseriesclassification.\+com/description.\+php?\+Dataset=\+FordB}} 