\chapter{Insect\+Wingbeat\+Sound}
\hypertarget{md_external_2data_2UCRArchive__2018_2InsectWingbeatSound_2README}{}\label{md_external_2data_2UCRArchive__2018_2InsectWingbeatSound_2README}\index{InsectWingbeatSound@{InsectWingbeatSound}}
\label{md_external_2data_2UCRArchive__2018_2InsectWingbeatSound_2README_autotoc_md131}%
\Hypertarget{md_external_2data_2UCRArchive__2018_2InsectWingbeatSound_2README_autotoc_md131}%
 The data were generated by the UCR computational entomology group and used in \mbox{[}1\mbox{]}. The data are power spectrum of the sound of insects passing through a sensor. The 11 classes are male and female mosquitos (Ae. aegypti, Cx. tarsalis, Cx. quinquefasciants, Cx. stigmatosoma), two types of flies (Musca domestica and Drosophila simulans) and other insects. More details about the conditions under which the data were collected are available in \mbox{[}1\mbox{]} and \mbox{[}2\mbox{]}.

Train size\+: 220

Test size\+: 1980

Missing value\+: No

Number of classses\+: 11

Time series length\+: 256

Data donated by Yanping Chen and Eamonn Keogh (see \mbox{[}1\mbox{]}, \mbox{[}2\mbox{]}, \mbox{[}3\mbox{]}).

\mbox{[}1\mbox{]} Chen, Yanping, et al. "{}\+Flying insect classification with inexpensive sensors."{} Journal of insect behavior 27.\+5 (2014)\+: 657-\/677.

\mbox{[}2\mbox{]} \href{https://sites.google.com/site/insectclassification/}{\texttt{ https\+://sites.\+google.\+com/site/insectclassification/}}

\mbox{[}3\mbox{]} \href{http://www.timeseriesclassification.com/description.php?Dataset=InsectWingbeatSound}{\texttt{ http\+://www.\+timeseriesclassification.\+com/description.\+php?\+Dataset=\+Insect\+Wingbeat\+Sound}} 