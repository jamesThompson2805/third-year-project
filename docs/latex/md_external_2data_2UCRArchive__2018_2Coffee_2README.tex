\chapter{Coffee}
\hypertarget{md_external_2data_2UCRArchive__2018_2Coffee_2README}{}\label{md_external_2data_2UCRArchive__2018_2Coffee_2README}\index{Coffee@{Coffee}}
\label{md_external_2data_2UCRArchive__2018_2Coffee_2README_autotoc_md35}%
\Hypertarget{md_external_2data_2UCRArchive__2018_2Coffee_2README_autotoc_md35}%
 Food spectrographs are used in chemometrics to classify food types, a task that has obvious applications in food safety and quality assurance. The coffee dataset is a two-\/class problem to distinguish between Robusta and Aribica coffee beans. Further information can be found in the original paper (see \mbox{[}1\mbox{]}). The data were first used in the time series classification literature by Bagnall et al. (see \mbox{[}2\mbox{]}).

Train size\+: 28

Test size\+: 28

Missing value\+: No

Number of classses\+: 2

Time series length\+: 286

Data donated by Katherine Kemsley and Anthony Bagnall (see \mbox{[}1\mbox{]}, \mbox{[}2\mbox{]}, \mbox{[}3\mbox{]}).

\mbox{[}1\mbox{]} Briandet, Romain, E. Katherine Kemsley, and Reginald H. Wilson. "{}\+Discrimination of Arabica and Robusta in instant coffee by Fourier transform infrared spectroscopy and chemometrics."{} Journal of agricultural and food chemistry 44.\+1 (1996)\+: 170-\/174.

\mbox{[}2\mbox{]} Bagnall, Anthony, et al. "{}\+Transformation based ensembles for time series classification."{} Proceedings of the 2012 SIAM international conference on data mining. Society for Industrial and Applied Mathematics, 2012.

\mbox{[}3\mbox{]} \href{http://www.timeseriesclassification.com/description.php?Dataset=Coffee}{\texttt{ http\+://www.\+timeseriesclassification.\+com/description.\+php?\+Dataset=\+Coffee}} 