\chapter{Adiac}
\hypertarget{md_external_2data_2UCRArchive__2018_2Adiac_2README}{}\label{md_external_2data_2UCRArchive__2018_2Adiac_2README}\index{Adiac@{Adiac}}
\label{md_external_2data_2UCRArchive__2018_2Adiac_2README_autotoc_md1}%
\Hypertarget{md_external_2data_2UCRArchive__2018_2Adiac_2README_autotoc_md1}%
 The Automatic Diatom Identification and Classification (ADIAC) project was a pilot study concerning automatic identification of diatoms (unicellular algae) on the basis of images. The data was donated by Andrei Jalba, a PhD student in the project, which was finished in the early 2000s. The outlines are extracted from thresholded images. Presumably the time series are generated as distance to a reference point (the centre being the obvious candidate). The data are very sinusoidal.

Train size\+: 390

Test size\+: 391

Missing value\+: No

Number of classses\+: 37

Time series length\+: 176

Data donated by Andrei Jalba (see \mbox{[}1\mbox{]}, \mbox{[}2\mbox{]}).

\mbox{[}1\mbox{]} Jalba, Andrei C., Michael HF Wilkinson, and Jos BTM Roerdink. "{}\+Automatic segmentation of diatom images for classification."{} Microscopy research and technique 65.\+1‐2 (2004)\+: 72-\/85.

\mbox{[}2\mbox{]} \href{http://www.timeseriesclassification.com/description.php?Dataset=Adiac}{\texttt{ http\+://www.\+timeseriesclassification.\+com/description.\+php?\+Dataset=\+Adiac}} 