\chapter{Olive\+Oil}
\hypertarget{md_external_2data_2UCRArchive__2018_2OliveOil_2README}{}\label{md_external_2data_2UCRArchive__2018_2OliveOil_2README}\index{OliveOil@{OliveOil}}
\label{md_external_2data_2UCRArchive__2018_2OliveOil_2README_autotoc_md158}%
\Hypertarget{md_external_2data_2UCRArchive__2018_2OliveOil_2README_autotoc_md158}%
 Food spectrographs are used in chemometrics to classify food types, a task that has obvious applications in food safety and quality assurance. Each class of this data set is an extra virgin olive oil from alternative countries. Further information can be found in the original paper \mbox{[}1\mbox{]}. The data were first used in the time series classification literature in \mbox{[}2\mbox{]}.

Train size\+: 30

Test size\+: 30

Missing value\+: No

Number of classses\+: 4

Time series length\+: 570

Data donated by Katherine Kemsley and Anthony Bagnall (see \mbox{[}2\mbox{]}, \mbox{[}3\mbox{]}).

\mbox{[}1\mbox{]} Tapp, Henri S., Marianne Defernez, and E. Katherine Kemsley. "{}\+FTIR spectroscopy and multivariate analysis can distinguish the geographic origin of extra virgin olive oils."{} Journal of agricultural and food chemistry 51.\+21 (2003)\+: 6110-\/6115.

\mbox{[}2\mbox{]} Bagnall, Anthony, et al. "{}\+Transformation based ensembles for time series classification."{} Proceedings of the 2012 SIAM international conference on data mining. Society for Industrial and Applied Mathematics, 2012.

\mbox{[}3\mbox{]} \href{http://www.timeseriesclassification.com/description.php?Dataset=OliveOil}{\texttt{ http\+://www.\+timeseriesclassification.\+com/description.\+php?\+Dataset=\+Olive\+Oil}} 